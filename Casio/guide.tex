% mental note to run makeindex on final compilation

\documentclass[a5paper]{memoir}

\usepackage[english]{babel}
\usepackage[autostyle, english = american]{csquotes}
\usepackage{parskip}
\usepackage{makeidx}
\usepackage{amssymb}
\usepackage{amsmath}
\usepackage{tabu}
\usepackage{tabularx}
\usepackage{systeme}
\usepackage{graphicx}
\usepackage[margin=1cm]{caption}
\usepackage[super]{nth}
\usepackage[usenames,dvipsnames,svgnames,table]{xcolor}
\usepackage{hyperref}

\hypersetup{pdftitle={A Comprehensive Guide to the CASIO fx-9860GIIs},pdfauthor={Sun Yudong, Li Yicheng},bookmarksnumbered=true,bookmarksopen=true,bookmarksopenlevel=1,	colorlinks=true,linkcolor=black,pdfpagemode=UseOutlines,pdfpagelayout=TwoPageRight}

\MakeOuterQuote{"}
\makeindex

\newenvironment{acknowledgements}%
{\cleardoublepage\thispagestyle{empty}\null\vfill\begin{center}%
		\bfseries Acknowledgements\end{center}}%
{\vfill\cleardoublepage\null}

%\graphicspath{{subdir1/}{subdir2/}{subdir3/}...{subdirn/}}
\graphicspath{{screenshots/}}

% lambda
	% http://tex.stackexchange.com/questions/290165/how-do-i-get-a-nicer-lambda
	% ref.: egreg at http://tex.stackexchange.com/a/14406/
	\usepackage{pifont}
	\makeatletter
	\newcommand\Pimathsymbol[3][\mathord]{%
		#1{\@Pimathsymbol{#2}{#3}}}
	\def\@Pimathsymbol#1#2{\mathchoice
		{\@Pim@thsymbol{#1}{#2}\tf@size}
		{\@Pim@thsymbol{#1}{#2}\tf@size}
		{\@Pim@thsymbol{#1}{#2}\sf@size}
		{\@Pim@thsymbol{#1}{#2}\ssf@size}}
	\def\@Pim@thsymbol#1#2#3{%
		\mbox{\fontsize{#3}{#3}\Pisymbol{#1}{#2}}}
	\makeatother
	% the next two lines are needed to avoid LaTeX substituting upright from another font
	\input{utxmia.fd}
	\DeclareFontShape{U}{txmia}{m}{n}{<->ssub * txmia/m/it}{}
	% you may also want
	\DeclareFontShape{U}{txmia}{bx}{n}{<->ssub * txmia/bx/it}{}
	% just in case
	%\DeclareFontShape{U}{txmia}{l}{n}{<->ssub * txmia/l/it}{}
	%\DeclareFontShape{U}{txmia}{b}{n}{<->ssub * txmia/b/it}{}
	% plus info from Alan Munn at http://tex.stackexchange.com/questions/290165/how-do-i-get-a-nicer-lambda?noredirect=1#comment702120_290165
	\newcommand{\pilambdaup}{\Pimathsymbol[\mathord]{txmia}{21}}

% linear algebra
	% http://tex.stackexchange.com/questions/2233/whats-the-best-way-make-an-augmented-coefficient-matrix
	\newenvironment{amatrix}[1]{%
		\left(\hspace{0.15cm}\begin{array}{@{}*{#1}{c}|c@{}}
		}{%
	\end{array}\hspace{0.15cm}\right)
	}
	
	\def\VR{\kern-\arraycolsep\strut\vrule &\kern-\arraycolsep}
	\def\vr{\kern-\arraycolsep & \kern-\arraycolsep}

\pretitle{\begin{center}\Huge\bfseries}
	\title{A Comprehensive Guide to the CASIO fx-9860GIIs}
	\posttitle{\par\vskip1em{\normalfont\normalsize\scshape The better graphing calculator for H2 Mathematics \\ \vspace{1cm} Quick Reference Math included\par\vfill}\end{center}}
\author{Sun Yudong, Li Yicheng \\ 15S6G \\ Hwa Chong Institution (College Section)}
\predate{\vfill\begin{center}\large}

\def\code#1{\texttt{#1}}
\def\note#1{\textcolor[HTML]{109fa9}{\textbf{\textit{Note:}}} #1}
\def\sidenote#1{\textcolor[HTML]{00559e}{\textbf{\textit{Sidenote:}}} \textcolor[HTML]{3c90d8}{#1}}
\def\tip#1{\textcolor[HTML]{a9109f}{\textbf{\textit{Tip:}}} #1}
\def\caution#1{\textcolor[HTML]{b02f10}{\textbf{\textit{Caution:}}} #1}
\def\example#1{{\centering -- \textcolor[HTML]{2a6c06}{\textbf{\textit{Example:}}} #1 --\par}}

\def\runmat{\code{RUN$\cdot$MAT} }

% define column vector
% creds: http://tex.stackexchange.com/questions/2705/typesetting-column-vector#2712
% usage: \colvec{m}{a}{b}{c}
\newcount\colveccount
\newcommand*\colvec[1]{
	\global\colveccount#1
	\left(
	\begin{array}{c}
	% \begin{pmatrix}
		\colvecnext
	}
	\def\colvecnext#1{
		#1
		\global\advance\colveccount-1
		\ifnum\colveccount>0
		\\
		\expandafter\colvecnext
		\else
	%\end{pmatrix}
	\end{array}
	\right)
	\fi
}

%define function buttons
\def\Fone{(\code{F1}) }
\def\Ftwo{(\code{F2}) }
\def\Fthree{(\code{F3}) }
\def\Ffour{(\code{F4}) }
\def\Ffive{(\code{F5}) }
\def\Fsix{(\code{F6}) }

\def\fone{(\code{F1})}
\def\ftwo{(\code{F2})}
\def\fthree{(\code{F3})}
\def\ffour{(\code{F4})}
\def\ffive{(\code{F5})}
\def\fsix{(\code{F6})}

\setlength{\parindent}{0pt}
\setlength{\parskip}{1ex plus 0.5ex minus 0.2ex}
\setlength{\footnotesep}{\baselineskip}

%\setcounter{tocdepth}{2} % which level the sectioning commands are printed in the ToC
%\setcounter{secnumdepth}{4} %up to what level the sectioning titles are numbered\makeatletter
\setsecnumdepth{subsection}

\newcommand{\specialcell}[2][c]{%
	\begin{tabular}[#1]{@{}c@{}}#2\end{tabular}}
% \specialcell{No. of serious\\accidents in a day}

\newcolumntype{L}[1]{>{\raggedright\let\newline\\\arraybackslash\hspace{0pt}}m{#1}}
\newcolumntype{C}[1]{>{\centering\let\newline\\\arraybackslash\hspace{0pt}}m{#1}}
\newcolumntype{R}[1]{>{\raggedleft\let\newline\\\arraybackslash\hspace{0pt}}m{#1}}

\newcommand{\addtoindex}[1]{#1\index{#1}}

\begin{document}

\begin{titlingpage}
	\maketitle
\end{titlingpage}

\frontmatter

\begin{acknowledgements}
	Were we to delve deeper into die Welt der Mathematik, we would scarce be able to differentiate the real and the imaginary.
	
	Who would have known, that with functions ranging limitless, and planes stretching to infinity, we could prosper in this World of Mathematics. Just as a plane cannot be fully defined without its normal vector, we thank our tutor Mr Yee Weng Hong for his patient direction, without whom this guide would not have been possible. 
	
	We also thank the Hwa Chong Institution Mathematics Department, for it has been the distribution of their awesome and systematic notes, which we have sampled very generously in this friendly guide, that has greatly increased our probability of success.
	
	And to our friends, with whom our paths have crossed, thank you for supporting us the whole way.
	
	This integrated guide is dedicated to everyone who has dared to think different.
	
	Who, against all odds, shall stand firm against the evil Alevelus Examinus -- with their Casio fx-9860GIIs calculator in hand.
\end{acknowledgements}

\tableofcontents

\chapter{Preface}
This guide is provided as-is and was compiled by a bunch of students in 2016. We tried our best to be mathematically and technically accurate. But software versions (Ver 2.04 now) do change, and this guide may not be valid indefinitely. However, the source of this file is publically available on GitHub. Feel free to contact us.

If there are anything you want to do, but you can't seem to know how to, try Google, or RTFM. Otherwise, just experiment a bit. The CASIO interface should be intuitive enough.

Do note that some notes inside this guide were taken from the H2 Mathematics notes provided by the Hwa Chong Institution (College Section) Math Department. 

\vspace{0.7cm}

\begin{tabular}{l l}
	Sun Yudong: & sunjerry019 [at] gmail [dot] com \\
	Li Yicheng: & liyicheng340 [at] gmail [dot] com 
\end{tabular}


\mainmatter
\chapter{The Basics}
Every command you will ever need is organized neatly in the \code{[OPTN]} button on your graphing calculator (GC).

Unfortunately there are some areas where the CASIO calculator isn't very intuitive and we need some documentation.

\tip{You can just press the button that corresponds to the number/letter at the bottom right of the icons in the menu:}

\begin{figure}[h]
	\centering
	\includegraphics[width=0.5\textwidth]{mainmenu}
\end{figure}

\section{Display Modes}
As a special mention that you probably did not notice, we shall first introduce the different display modes that the calculator is capable of. In the Set Up menu (\code{[SHIFT] + [MENU]}), there are different display options, and we summarize their corresponding functions as follows:
\begin{center}
	\renewcommand{\arraystretch}{1.3}
	\begin{tabular}{|c|p{7.5cm}|}
		\hline
		\textbf{Mode}	&	\textbf{What it does} \\
		\hline
		\code{Fix}$n$	&	Rounds off answer to $n$ decimal places \\
		\hline
		\code{Sci}$n$	&	Rounds off answer to $n$ significant figures \\
		\hline
		\code{Norm}		&	Converts answer to scientific notation when the value $x$ lies outside of the following range: \vspace{2mm} \newline \hspace*{5mm} \textbf{\code{Norm1}}: $10^{-3}>|x|>10^{10}$ \newline \hspace*{5mm} \textbf{\code{Norm2}}: $10^{-9}>|x|>10^{10}$ \\
		\hline
	\end{tabular}
\end{center}

In addition to these options, there is also a \code{Eng} mode that adds on to those above mentioned modes. The \code{/E} indicator is on display while the engineering notation is in effect (e.g. \code{Norm1/E}). 

In the engineering mode, answers are displayed with standard order-of-magnitude prefixes, and the symbol that makes the mantissa a value from 1 to 1000 is automatically chosen:
\begin{center}
	\renewcommand{\arraystretch}{1.3}
	\begin{tabular}{|c|c||c|c|}
		\hline
		E (Exa)		& $\times10^{18}$	& m (milli)	& $\times10^{-3}$ \\
		\hline
		P (Peta)	& $\times10^{15}$	& $\mu$ (micro)	& $\times10^{-6}$ \\
		\hline
		T (Tera)	& $\times10^{12}$	& n (nano)	& $\times10^{-9}$ \\
		\hline
		G (Giga)	& $\times10^{9}$	& p (pico)	& $\times10^{-12}$ \\
		\hline
		M (Mega)	& $\times10^{6}$	& f (femto)	& $\times10^{-15}$ \\
		\hline
		k (kilo)	& $\times10^3$		& 	&  \\
		\hline
	\end{tabular}
\end{center}

You can also type these prefixes when doing calculations: they are available under \code{[OPTN] > $\triangleright\times$2 > ESYM} \Fone. You might find these extremely useful when doing calculations for Physics.

Under that same menu, there is also the \code{ENG} and $\overleftarrow{\texttt{ENG}}$ which shifts the decimal place of the displayed value 3 digits to the (left/right), and (decreases/increases) the exponent by 3 respectively.

\section{Radian/Degree/Gradian} \label{sec:angleunits}

Like the majority of other scientific calculators, this GC is able is accept input and display the magnitude of angles in either degrees, radians, or gradians. To change between the 3, enter the Set Up menu, and scroll down until you see a row labelled \code{Angle}. The 3 different units are the displayed at the bottom of the screen and can be selected by the \code{[F1]} to \code{[F3]} buttons.

Note that this setting is universal throughout your GC - it will change the units angles are handled in every mode including but not limited to usual calculations, graphing, etc, unless you specifically specify otherwise (not covered in this section). Therefore, it is vital that you check your units before undergoing any calculation, especially before and during examinations.

\section{Graphing} \label{sec:graphing}
To graph a function, go to \code{GRAPH} in the main menu. Enter the functions into \code{Y1} onwards. Press \code{DRAW} \Fsix or \code{[EXE]} to graph. Most of these are pretty intuitive, just some things you should take note:
\begin{itemize}
	\item The \code{Y} and \code{X}, etc. at the bottom of the screen is for entering the functions \code{Y1, Y2, \dots} and \code{X1, X2, \dots}. For the variable $x$, use the \code{[X,$\theta$,T]} button instead
	\item To restrict the domain of the function, type:
	\begin{center}
		\code{Y1 = $f(x)$,[start,end]}
	\end{center}
	\note{Restricting the domain disables the sketching of the inverse of a graph.}
\end{itemize}

\begin{figure}[h]
	\centering
	\includegraphics[width=0.5\textwidth]{sinx}
	\caption{Graph of sin($x$), for $-\pi \leq x \leq \pi$}
\end{figure}

After pressing \code{DRAW} (\code{F6} or \code{[EXE]}), you can press \code{[AC/on]} to break the plotting script.

To solve for anything, use \code{[G-Solv]} \Ffive.

\subsection{Sketching}
Once a graph is plotted, you can sketch additional graphs on top of the already plotted graph. This includes sketching the inverse\index{Sketching!Inverse} of a function, tangent\index{Sketching!Tangent} and normal\index{Sketching!Normal} to a point on the curve.

\begin{figure}[h]
	\centering
	\includegraphics[width=0.5\textwidth]{normtan_sinx}
	\caption{Graph of sin($x$), for $-\pi \leq x \leq \pi$, with the normal and tangent drawn at $x=-2.5$}
\end{figure}

All these are accessible under the Sketch menu (\code{SHIFT > F4}).

When sketching the normal/tangent, just press the number keys if you would like to enter a specific X-value to plot the normal/tangent at.

\subsection{Plotting a table from function} \label{sec:plottable}
To effectively use your calculator as a function generator, you can use the \code{TABLE} mode available in the menu:
\begin{enumerate}
	\item Go to \code{TABLE} and then enter your function just like how you would in \code{GRAPH}
	\item Go to \code{SET} \Ffive
\end{enumerate}

Once in table mode, you can enter values directly into the $X$ column to generate values using the function. 

This is especially useful when you are doing the same operation multiple times on different numbers, or when you are generating solving a Sequences and Series question using the Method of Differences.

\section{Solving for the roots of a polynomial}
There are mainly 2 ways to solve for the roots of a polynomial using the GC:
\begin{itemize}
	\item Using the \code{EQUA > Poly} \Ftwo app
	\item Plot a graph and solve for roots
\end{itemize}

The \code{EQUA > Poly} is quite intuitive, and similar to the standard issue CASIO fx-95SG scientific calculator, so we will just note some limitations/features of this GC:
\begin{itemize}
	\item The polynomial solver only accepts real coefficients
	\item You can change the \code{Set Up > Complex Mode} setting to \code{$a+bi$} for imaginary roots (but not imaginary coefficients)
\end{itemize}

To plot a graph, you go to \code{Graph}. Refer to Section \ref{sec:graphing} (Graphing) for more information.

\section{Solving a 1-variable Equation}
There are mainly 2 ways to solve a 1-variable equation (e.g. $e^x + 5x = 1$) using the GC:
\begin{itemize}
	\item Using the \code{EQUA > Solver} \Fthree app
	\item Using \code{Solve} or \code{SolveN}
	\item Plotting a graph
	\begin{itemize}
		\item Move everything to one side and solve for root
		\item Solve for the intersections of 2 or more graphs
	\end{itemize}
\end{itemize}

\subsection{Solving using \code{EQUA > Solver}}
To use the built-in equation solver:
\begin{enumerate}
	\item Go to \code{EQUA > Solver} \Fthree
	\item Enter the equation you would like to solve under \code{Eq}. Remember to put the = sign.
	\begin{itemize}
		\item Alternatively, you can use \code{RCL} \Fone to recall functions entered into \code{GRAPH} or \code{TABLE}
	\end{itemize}
	\item Press \code{[EXE]} and put your initial guess under \code{X}. An example can be as such:
	\begin{center}
		\code{
			\begin{tabular}{|r|l|}
				\hline
				X 		& 0 \\
				\hline
				Lower 	& -9E+99 \\
				\hline
				Upper	& 9E+99 \\
				\hline
			\end{tabular}
		}
	\end{center}
	\item Press \code{Solv} \Fsix	
\end{enumerate}

\note{}The initial guess isn't very important for equations with unique solutions. However, if the equation has multiple solutions, it will give the answer closer to the original guess.

For example, given the equation $(x-1)(x-3)=0$, entering the following will give the corresponding answers:
\begin{center}
	\renewcommand{\arraystretch}{1.2}
	\begin{tabular}{|c|c|}
		\hline
		$X$ 	& Answer \\
		\hline
		0		& 1 \\
		\hline
		1.5 	& 1 \\
		\hline
		2		& 1 \\
		\hline
		2.5		& 3 \\
		\hline
		4		& 3 \\
		\hline
	\end{tabular}
\end{center}

After solving the equation, you can always access the value again by going to \runmat and typing \code{X}.

\subsection{Solving using \code{Solve} or \code{SolveN}}
Alternatively, you can use the \code{Solve} or \code{SolveN} functions in \runmat to obtain the solution. They can be found under \code{[OPTN] > CALC}\Ffour.

Their usage is as follows, as far as we have discovered:
\begin{center}
	\code{Solve(<equation>,<initial guess>)} \\
	\code{SolveN(<equation>)}
\end{center}

Both \code{Solve} and \code{SolveN} will give you only 1 answer, and works similar to how the \code{EQUA > Solver} works.

\section{Solving a Multi-variable Linear System}
There are 3 types of linear systems, namely systems with:
\begin{itemize}
	\item No solution
	\item 1 unique solution
	\item Infinitely many solutions
\end{itemize}

To solve a multi-variable system of linear equations using your GC, there are 2 main ways:
\begin{enumerate}
	\item Using the \code{EQUA > Siml} \Fone app
	\item Using matrices
\end{enumerate}

\subsection{Using the \code{EQUA > Siml} app}
While you can use matrices to solve any kind of linear system, you can use the \code{EQUA > Siml} app \textit{\textbf{only if the linear system has a unique solution}}.

\note{Should you try to use the \code{EQUA > Siml} app to solve for a linear system with no solution, or infinitely many solutions, the calculator would just throw a \code{Ma Error} at you while you stare, bemused, wondering why the calculator is so bad to you.}

Anyhow, the app is intuitive enough, so I shall skip it.

\subsection{Using Matrices} \label{sec:usingMatrices}

To solve a linear system the matrix way, let's first consider the following linear system:
\begin{center}
	\syslineskipcoeff{1.2}
	\systeme{
		\hspace{0.2cm} x+y+2z=9,
		\hspace{0.2cm} 2x+4y-3z=1,
		\hspace{0.2cm} 3x+6y-5z=0
	}
\end{center}

By taking the coefficients of the variables $x$, $y$ and $z$, we can form the following augmented matrix:
\[
	\begin{amatrix}{3}
		1 & 1 & 2  & 9 \\  
		2 & 4 & -3 & 1 \\
		3 & 6 & -5 & 0
	\end{amatrix}
\]

To solve for $x$, $y$ and $z$:\footnote{ Fn Keys have been left out to save space.}
\begin{enumerate}
	\item Go to \runmat \code{> $\triangleright$MAT}
	\item Select \code{Mat A} and enter the dimensions
	\begin{itemize}
		\item $m$ is the number of rows, in this case 3
		\item $n$ is the number of columns, in this case 4
	\end{itemize}
	\item Enter the augmented matrix above accordingly, putting the augmented column into the 4th column:
	\begin{equation*}
		\left[
			\begin{array}{cccc}
			1 & 1 & 2  & 9 \\  
			2 & 4 & -3 & 1 \\
			3 & 6 & -5 & 0
			\end{array}
		\right]
	\end{equation*}
	\item Press \code{[EXIT]} then \code{[EXIT]} to go back to the main \runmat screen
	\item Go to \code{[OPTN] > MAT > $\triangleright$ > Rref}
	\item Then \code{ $\triangleright \times 3$ > Mat}, then \code{[ALPHA] + A}, giving the following:
	\begin{center}
		\code{Rref Mat A}
	\end{center}
	\item Press \code{[EXE]}, and then you should be able to get the following matrix:
	\begin{equation*}
		\left[
			\begin{array}{cccc}
			1 & 0 & 0 & 1 \\  
			0 & 1 & 0 & 2 \\
			0 & 0 & 1 & 3
			\end{array}
		\right]
	\end{equation*}
\end{enumerate}

This can then be rewritten as an augmented matrix:
\begin{equation*}
\bordermatrix{  & x & y & z & \vr ~ \cr
			  ~ & 1 & 0 & 0 & \VR 1 \cr
			  ~ & 0 & 1 & 0 & \VR 2 \cr
			  ~ & 0 & 0 & 1 & \VR 3 }
\end{equation*}

which can then be rewritten as:
\begin{center}
	\syslineskipcoeff{1.2}
	\systeme{
		\hspace{0.2cm} x=1,
		\hspace{0.2cm} y=2,
		\hspace{0.2cm} z=3
	}
\end{center}

Notice that the left side of the final augmented matrix is the identity matrix. This is what you would typically see for a linear system with a unique solution.

This same procedure can be used to find the solutions of any linear system with any number of variables.

The following are possible matrices you can obtain after the \code{Rref} command for the other types of linear systems:

\begin{itemize}
	\item \textbf{No solution}
	\begin{equation*}
		\begin{tabu}{ccc}
			\begin{amatrix}{3}
			1 & 0 & 2 & 2 \\  
			0 & 1 & 3 & 3 \\
			0 & 0 & 0 & 1
			\end{amatrix} & \rightarrow & \syslineskipcoeff{1.2}
			\systeme{
				\hspace{0.2cm} x=2-2t,
				\hspace{0.2cm} y=3-3t,
				\hspace{0.2cm} 0z=1
			}\\
		\end{tabu}
	\end{equation*}
	where $t \in \mathbb{R}$
	
	Since $0=1$ is not possible, the above linear system is inconsistent and has no solutions.
	\item \textbf{Infinitely many solutions}
	\begin{equation*}
		\begin{tabu}{ccc}
			\begin{amatrix}{3}
			1 & 0 & 2 & 2 \\  
			0 & 1 & 3 & 3 \\
			0 & 0 & 0 & 0
			\end{amatrix} & \rightarrow & \syslineskipcoeff{1.2}
			\systeme{
				\hspace{0.2cm} x=2-2t,
				\hspace{0.2cm} y=3-3t,
				\hspace{0.2cm} z=t
			}\\
		\end{tabu}
	\end{equation*}
	where $t \in \mathbb{R}$
	
	In this case, the answer is obtained easily through back-substitution. Rewriting the augmented matrix, we observe that:
	\begin{center}
		\syslineskipcoeff{1.2}
		\systeme{
			\hspace{0.2cm} x+2z=2,
			\hspace{0.2cm} y+3z=3
		}
	\end{center}
	
	By letting $z$ be a parameter $t$ ($t \in \mathbb{R}$), we observe that:
	\begin{equation*}
	\begin{tabu}{ccc}
		\syslineskipcoeff{1.2}
		\systeme{
			\hspace{0.2cm} x+2t=2,
			\hspace{0.2cm} y+3t=3,
			\hspace{0.2cm} z=t
		} & \rightarrow & \syslineskipcoeff{1.2}
			\systeme{
				\hspace{0.2cm} x=2-2t,
				\hspace{0.2cm} y=3-3t,
				\hspace{0.2cm} z=t
			}\\
		\end{tabu}
	\end{equation*}
	
	Thus obtaining the general solution. 
\end{itemize}

You will find this useful for both "System of Linear Equations" and "Vectors" (See Chapter \ref{chap:vectors}). 

For a more in-depth understanding of what is going on here, you may wish to refer to Section \ref{sec:solvLinSys}.

\section{Taking Integrals}
\subsection{Definite Integrals}
Your calculator can be used to solve definite integrals.

One way is to find the area under the graph after plotting the equation.

\begin{enumerate}
	\item Plot the graph following the instructions from Section \ref{sec:graphing}.
	\item Press \code{G-Solv} \Ffive, followed by $\triangleright$ \Fsix and \code{$\int$dx} \Fthree.
	\item A cursor on the curve will now appear on the graph, along with $x$ and $y$ values. Move the cursor to the lower bound of the definite integral you're trying to find, and press \code{EXE}, followed by the upper bound, and press \code{EXE} again.
	\item The area under the graph should now be shaded and the value of the area is displayed as \code{$\int$dx}.
\end{enumerate}

This method is great if you already have the graph plotted and want to find the integral quickly without entering the equation again. If you do not need to plot the graph and simply want to find the definite integral directly, you may find the integration function in the \runmat app more useful.

\begin{enumerate}
	\item From the main menu, enter the \runmat app (\code{1}).
	\item Press the \code{OPTN} button and select the \code{CALC} option \Ffour.
	\item Under the \code{CALC} submenu, choose the \code{$\int$dx} option (\code{F4}).
	\item Enter the equation as per standard Mathematical notation. 
	
	\note{You should change the variable to be integrated from whatever letter it was originally to $x$ since the calculator forces the integral to be integrated with respect to $x$ as can be seen by the $dx$ at the back.}
\end{enumerate}

\subsection{Indefinite Integrals}
SEAB bans all calculators capable of direct symbolic integration and differentiation. In other words, you cannot enter $3x^2$ into your graphing calculator and get $x^3$ or vice versa. To be fair, this feature was never in the international version of the Casio anyway.

Thankfully, even in the Singaporean version of the Casio, you can plot the graph of the integral visually.

For example, we will try to plot the integral of $2x$. Assuming that your GC is working properly and the Laws of Mathematics have not changed in drastic ways, the graph should be identical to that of $x^2$.

\begin{enumerate}
	\item From the main menu, enter the \code{GRAPH} app (\code{5}).
	\item Under \code{Y1}, enter the function $2x$.
	\item Move to \code{Y2}. Enter the following function:
	\begin{equation*}
	\code{Y2}=\int_{0}^{x} \code{Y}1 \, dx
	\end{equation*}
	\note{\code{Y2} will take a significant amount of time to plot (around 5-10 seconds). It is recommended that you deselect \code{Y1} by pressing \code{SEL} \Fone to prevent \code{Y1} from being plotted, saving a couple of seconds.}
	\item Press \code{DRAW} \fsix. A parabola that identical to $y=x^2$ should be yield. You can plot $y=x^2$ on top of the integral to confirm.
\end{enumerate}

While this function is unable to give you the equation of the integral directly, you can use double check your integrals using this method. Plotting the integral using this method can also be useful if you have to find multiple definite integrals (using the trace function in the graph page).

\chapter{Vectors} \label{chap:vectors}

Casio added support for vectors with the \code{02.04.5301} firmware update. This is added on top of support for Matrices, which will be covered mainly in Chapter \ref{chap:linalge} (Linear Algebra).\footnote{While vectors are also covered in the Linear Algebra syllabus, this chapter will focus more on what is needed to tackle the H2 syllabus.}

\caution{Vector capabilities will be disabled in the new \textit{Exam Mode} that ships with the Ver. \code{2.09} update.\footnote{We will not be covering this as we are covering Ver. \code{2.04.5301}, the latest SEAB approved firmware version.}} However, as of present we are not aware of any requirements to activate \textit{Exam Mode} during SEAB's examinations.

\section{Entering a Vector}

Functionally, vectors in this GC is no different from a 1 by $n$ or a $n$ by 1 matrix. Hence, you can use \code{Mat} and \code{Vct} interchangeably, so long as they are of the correct order.

To enter a vector into the GC:

\begin{enumerate}
	\item Enter the \runmat mode and select \code{$\triangleright$MAT} \fthree
	\item Select \code{M$\leftrightarrow$V} \fsix. The top row should now say \code{Vector} and the \code{Vct A} row should be highlighted by default
	\item Press \code{EXE} and enter the vector's dimensions. For the following example vector, it is a $3\times1$ vector, so enter $m=3$ and $n=1$
	\begin{equation*}
		\colvec{3}{3}{1}{4}
	\end{equation*}
	\item Enter the values in your vector into the three rows as shown on your textbook/exam paper
	\item Press \code{EXIT} twice to return to the main \runmat screen
\end{enumerate}

Alternatively, if you are in \runmat and your Input/Output is set to \code{Math} in the settings, you can enter vectors simply by going to \code{MATH} \Ffour \code{>} \code{MAT} and then choosing the appropriately sized matrix. These entered matrices will replace your \code{Vct A}/etc. described in the next few sections.

Note that the labelling follows the conventional $m \times n$ format, where $m$ is the number of rows, and $n$ is the number of columns.

\section{Vector Calculations}

\subsection{Addition and Subtraction}

To perform addition or subtraction calculations between two vectors:

\begin{enumerate}
	\item At the main \runmat screen, press \code{[OPTN]}, followed by \code{MAT} \ftwo. Press \code{$\triangleright$} \Fsix twice, then select \code{Vct} \fone.
	\item Press the \code{[ALPHA]} button, then press the corresponding letter that represents your first vector. For example, if you entered the vector at the first row previously, press the button that corresponds to \code{A} (which is \code{[X,$\theta$,T]}).
	\item Press the usual addition or subtraction signs at the bottom right of your calculator depending on which calculation you wish to do.
	\item Repeat steps 1 and 2 for your second vector, then press \code{[EXE]} to calculate the result.
\end{enumerate}

\subsection{Dot Product}

Vector specific calculations are slightly more complicated than addition or subtraction. To find the dot product of two vectors:

\begin{enumerate}
	\item In the vector section under \code{OPTN > MAT}, select \code{DotP} \ftwo.
	\item Press \code{Vct}, then press the corresponding letter that represents the first vector in your dot product expression.
	\item Press the \code{[,]} button, then repeat step 2 but for the second vector in your dot product expression. It is good practice at this stage to press the \code{[)]} button to close the opening bracket that comes together with the \code{DotP(} function.
	\item Press \code{[EXE]} to calculate the result.
\end{enumerate}

\subsection{Cross Product}

Cross product calculations are similar to dot products. To find the cross product between two vectors:

\begin{enumerate}
	\item In the vector section under \code{OPTN > MAT}, select \code{CrsP} \fthree.
	\item Press \code{Vct}, then press the corresponding letter that represents the first vector in your cross product expression.
	\item Press the \code{[,]} button, then repeat step 2 but for the second vector in your cross product expression. It is good practice at this stage to press the \code{[)]} button to close the opening bracket that comes together with the \code{CrossP(} function.
	\item Press \code{[EXE]} to calculate the result.
\end{enumerate}

It is important to note here that the \code{[$\times$]} button on your calculator will not work as a cross or dot product operator! Should you attempt to use it as such, your GC will return a \code{Dimension ERROR}.

\subsection{Angle Between Two Vectors}

Your GC is also capable of finding the angle between two vectors. To do so:

\begin{enumerate}
	\item In the vector section under \code{OPTN > MAT}, select \code{Angle} \ffour.
	\item Press \code{Vct}, then press the corresponding letter that represents the first vector.
	\item Press the \code{[,]} button, then repeat step 2 but for the second vector. It is good practice at this stage to press the \code{[)]} button to close the opening bracket that comes together with the \code{Angle(} function.
	\item Press \code{[EXE]} to calculate the result. Note that the angle calculated can be in radians, degrees, or gradians depending on your settings at the moment. Refer to Section \ref{sec:angleunits} if you are not sure how to change the units.
\end{enumerate}

\subsection{Unit Vectors}

If you ever need to find the unit vector of a particular vector, you can do so directly with your GC. To do so:

\begin{enumerate}
	\item In the vector section under \code{OPTN > MAT}, select \code{UntV} \ffive.
	\item Press \code{Vct}, then press the corresponding letter that represents your vector. It is good practice at this stage to press the \code{[)]} button to close the opening bracket that comes together with the \code{UnitV(} function.
	\item Press \code{[EXE]} to calculate the result.
\end{enumerate}

\section{Problems involving Vectors}
\subsection{Finding the line of intersection between 2 or more planes}
Let's say you are given 2 planes:

\begin{equation*}
	\pi_{1}: \textbf{r} \cdot \colvec{3}{3}{1}{5} = 4 \hspace{5mm} \text{ and } \hspace{5mm} \pi_{2}: \textbf{r} \cdot \colvec{3}{5}{1}{3} = 2
\end{equation*}

First, convert them to linear equations:
\begin{equation*}
\syslineskipcoeff{1.2}
\systeme{
	\hspace{0.2cm} 3x+y+5z=4,
	\hspace{0.2cm} 5x+y+3z=2
}
\end{equation*}

Then solve for the general solution as described in Section \ref{sec:usingMatrices} (Using Matrices to Solve Multi-variable Linear System):
\begin{equation*}
\syslineskipcoeff{1.2}
\systeme{
	\hspace{0.2cm} x=t-1,
	\hspace{0.2cm} y=-8t+7,
	\hspace{0.2cm} z=t
}\hspace{5mm} \text{where } t \in \mathbb{R}
\end{equation*}

Separating out the $t$, we realize that the solution (the line of intersection) can be written as such:
\begin{equation*}
\colvec{3}{x}{y}{z} = \textbf{r} = \colvec{3}{1}{-8}{1}t + \colvec{3}{-1}{7}{0} \hspace{5mm} t \in \mathbb{R}
\end{equation*}

Similarly, you can expand this method to find the point of intersection (unique solution). 

\chapter{Statistics}
One major use of a graphing calculator is for use in statistics. In the following chapters, we will outline the methods with which we can use our GC. Calculator functions in this section can generally be found under \code{[OPTN] > [STAT]}. Permutation and Combinations will not be covered. Calculated variables can be found under \code{[VARS] > [STAT]}

In this chapter, all random variables are represented by capital letters (usually $X$ or $Y$). Small letters are reserved to represent values (e.g. $x$).

\note{However, please do not use $Z$, as it is reserved for the standard normal distribution.}

\section{Discrete Random Variable}
A discrete (countable) random variable $X \in \mathbb{Z}^*$ has expectation (or expected value or mean) $\mu$
\begin{equation}
	\mathrm{E}(X)=\mu=\sum_{\mathrm{all}~x}^{}x\mathrm{P}(X=x)
\end{equation}
(You can think of this as the weighted sum of all the possibilities for $X$)

and variance $\sigma^2=\mathrm{stddev}^2$
\begin{equation}
	\mathrm{Var}(X)=\sigma^2=\mathrm{E}(X-\mu)^2=\mathrm{E}(X^2)-\mu^2
\end{equation}
\note{The above does not apply to continuous random variables.}

The function that gives the probability that a discrete random variable is exactly equal to some value is called the \textit{Probability Mass Function} (PMF). However, we will write PDF throughout this section instead, so as to be in-line with the curriculum/ calculator functions.

\subsection{Using Data}
If given a set of histogram data (i.e. categories and its corresponding frequencies), one can calculate statistical properties of it.

For example, given the following data:
$$
\setlength\tabulinesep{2mm}
\begin{tabu}{ |c|c|c|c|c| }
	\hline
	x & 0 & 1 & 2 & 3 \\
	\hline 
	\mathrm{P}(X=x)  & \frac{1}{8}  & \frac{3}{8}  & \frac{3}{8} &\frac{1}{8}  \\
	\hline
\end{tabu}
$$

You can do 1-Variable Statistical calculations by:
\begin{enumerate}
	\item Go into \code{STAT} and then enter the X-values into \code{List 1} and its frequency into \code{List 2}
	\item Go to \code{CALC > SET} and adjust the settings accordingly for 1-Var (\code{XList:List1, Freq:List2})
	\item \code{[EXIT]} and then choose \code{1VAR}
\end{enumerate} 

If you want to further manipulate these calculated values (such as squaring $\sigma$ to find variance), you can go to \runmat and then \code{[VARS] > [STAT] > [X]} to finding the value(s) you wanted.

Let's do a simple example to illustrate how to do this:

\example{Expressway accidents}

Over a period of 120 days, the number of serious accidents in a day along a certain expressway are recorded as follows:

\begin{center}
	\renewcommand*{\arraystretch}{1.2}
	\begin{tabular}{|C{0.5cm}|C{0.5cm}|C{0.5cm}|C{0.5cm}|C{0.5cm}|C{0.5cm}|C{0.5cm}|c|}
		\hline
		0 & 1 & 2 & 3 & 4 & 5 & 6 & 7 or more \\
		\hline 
		21 & 36 & 32 & 18 & 9 & 3 & 1 & 0  \\
		\hline
	\end{tabular}
\end{center}

where the \nth{1} row is the number of serious accidents in a day, and the \nth{2} is the number of days that number of serious accidents was recorded on.

To find the mean and variance of this dataset:
\begin{enumerate}
	\item Go to \code{STAT} and enter the data appropriately:
	\begin{figure}[h]
		\centering
		\includegraphics[width=0.5\textwidth]{expressway_accident_1}
	\end{figure}
	\item Go to \code{CALC > SET} and set the \code{1Var XList} as \code{List1} and \code{1Var Freq} as \code{List2}. You need not care about \code{2Var} for this example
	\item Press \code{[EXIT]} and then \code{1VAR} \Fone and you should get the following:
	\begin{figure}[h]
		\centering
		\includegraphics[width=0.5\textwidth]{expressway_accident_2}
	\end{figure}
	\item You now have the mean $\bar{x}$ and standard deviation $\sigma x$
	\item To find the variance of this data, go to \runmat and you will find $\sigma x$ under \code{VARS > STAT > X > $\sigma$x}. Enter that and then square it.	
\end{enumerate}

You have now found that $\bar{x} = 1.7583$ and $\text{Var}(x)=1.7499$. 

\sidenote{Since the variance and the mean is about the same, we can thus conclude that a Poisson model may be valid for this data set.}

Refer to Section \ref{sec:plotstat} (Plotting Statistics/Visualization) for plotting statistical data.

\subsection{Binomial Distribution} \label{sec:binom}
A binomial random variable $X$ has the following characteristics:
\begin{enumerate}
	\item The experiment consists of $n$ repeated independent trials
	\item Each trial only has 2 outcomes: "success" or "failure"
	\item The probability of a `success' $p$ is constant in each trial
\end{enumerate}

% The probability distribution of $X$ is called the \textit{binomial distribution}, and 
The probability $\textrm{P}(X = x)$ of obtaining $x$ successes in $n$ trials is given by:

\begin{equation}
	\mathrm{P}(X=x) = {n \choose x} p^x(1-p)^{n-x},~x \in \mathbb{Z}^*
\end{equation}

and has $\mathrm{E}(X) = np$ and $\mathrm{Var}(X)=np(1-p)$.

We write $X \sim \mathrm{B}(n, p)$.

\note{the probability is only non-zero when $0 \leq x \leq n$ (i.e. there is an upper bound).}

\subsubsection{Binomial PDF}

To obtain the probability $\textrm{P}(X = x)$, we can either use the formula, or use the in-built \code{\addtoindex{BinomialPD}} function:
\begin{center}
	\code{BinomialPD([X],$n$,$p$)}
\end{center}

When \code{X} is not provided, the output answer will be a \code{List}.

Alternatively, we can use the \code{STAT} app to obtain the probability:
\begin{enumerate}
	\item Go into \code{STAT} (\code{2} in the menu)
	\item Press \code{DIST}\Ffive\code{ > BINM}\Ffive\code{ > Bpd}\Fone
	\item Set the following:
	\begin{center}
		\setlength{\tabcolsep}{10pt}
		\renewcommand{\arraystretch}{1.1}
		\begin{tabular}{|l|l|}
			\hline
			Data		& Variable \\
			\hline
			$x$			& X \\
			\hline
			Numtrial	& $n$ \\
			\hline
			$p$			& $p$ \\
			\hline
			Save Res	& \code{None} \\
			\hline
		\end{tabular}
	\end{center}
	\item Press \code{[EXE]} or scroll to \code{Execute} and press \code{CALC}\Fone
	\item You can do further manipulation of the value obtained by:
	\begin{enumerate}
		\item Going to \runmat
		\item Go to \code{[VARS] > STAT}\Fthree\code{> RESLT}\Fsix\code{> DIST}\Fthree
		\item There you can find the $p$ value by pressing $p$ \Fone and then \code{[EXE]}
	\end{enumerate} 
\end{enumerate}

\subsubsection{Binomial CDF}

To obtain the probability $\textrm{P}(X \leq x)$ (\textit{cumulative distribution function} of $x$), we can use the in-built \code{\addtoindex{BinomialCD}} function:

\begin{center}
	\code{BinomialCD([X],$n$,$p$)}
\end{center}

Similarly, if \code{X} is not provided, the output answer will be a \code{List}.

Alternatively, following instructions in the previous section (Section \ref{sec:binom} $\Rightarrow$ Binomial PDF), and choosing \code{Bcd} instead. 

As a general rule of thumb:
\begin{center}
	\setlength{\tabcolsep}{10pt}
	\renewcommand{\arraystretch}{1.1}
	\begin{tabular}{|c|c|}
		\hline
		Answer in:		& Use: \\
		\hline
		$\mathbb{Z}$	& Table \\
		\hline
		$\mathbb{R}$	& Graph \\
		\hline
	\end{tabular}
\end{center}

Refer to Section \ref{sec:plottable} for plotting results from a table, which might be useful for plotting the PDF/CDF of a binomial distribution, since $\mathrm{P}(X=x)$ is undefined when $x \notin \mathbb{Z}^*$

\subsection{Poisson Distribution} \label{sec:poisson}
A random variable that follows a Poisson distribution has the following characteristics:
\begin{itemize}
	\item Events occur randomly in a fixed interval and independently of one another.
	\item The mean rate of occurrence of the event is \textbf{constant} in the given interval.
	\item The probability of more than 1 event occuring within a short interval is \textbf{negligible}. 
\end{itemize}

The probability $\textrm{P}(X = x)$ of obtaining $x$ successes in a fixed interval is given by:
\begin{equation}
	\textrm{P}(X=x)=e^{-\pilambdaup}\frac{\pilambdaup^x}{x!},~x \in \mathbb{Z}^*
\end{equation}

We write $X \sim \textrm{Po}(\pilambdaup)$, where $\pilambdaup$ is the mean number of occurrence of the event in that fixed interval. 

This random variable $X$ has $\mathrm{E}(X)=\mathrm{Var}(X)=\pilambdaup$.

\note{}
\begin{itemize}
	\item $X$ ranges over an infinite number of integer values (i.e. there is no upper bound)
	\item $\pilambdaup \propto t$, where $t$ is the length of the interval
\end{itemize}

\note{}
\begin{center}
	\setlength{\tabcolsep}{10pt}
	\renewcommand{\arraystretch}{1.1}
	\begin{tabular}{|c|c|}
		\hline 
		$\pilambdaup \in \mathbb{Z}$					& 2 modes\\
		\hline
		$\pilambdaup \in \mathbb{R} \notin \mathbb{Z}$	& 1 mode\\
		\hline
	\end{tabular}
\end{center}

\subsubsection{Poisson PDF}
To obtain the probability $\textrm{P}(X = x)$, we can either use the formula, or use the in-built \code{\addtoindex{PoissonPD}} function:
\begin{center}
	\code{PoissonPD(x,$\pilambdaup$)}
\end{center}

Alternatively, we can use the \code{STAT} app to obtain the probability:
\begin{enumerate}
	\item Go into \code{STAT} (\code{2} in the menu)
	\item Press \code{DIST}\Ffive\code{ > POISN}(\code{$\triangleright$ F1})\code{ > Ppd}\Fone
	\item Set the following:
	\begin{center}
		\setlength{\tabcolsep}{10pt}
		\renewcommand{\arraystretch}{1.2}
		\begin{tabular}{|l|l|}
			\hline
			Data		& Variable \\
			\hline
			$x$			& X \\
			\hline
			$\mu$		& $\pilambdaup$ \\
			\hline
			Save Res	& \code{None} \\
			\hline
		\end{tabular}
	\end{center}
	\item Press \code{[EXE]} or scroll to \code{Execute} and press \code{CALC}\Fone
	\item You can do further manipulation of the value obtained by:
	\begin{enumerate}
		\item Going to \runmat
		\item Go to \code{[VARS] > STAT}\Fthree\code{> RESLT}\Fsix\code{> DIST}\Fthree
		\item There you can find the $p$ value by pressing $p$ \Fone and then \code{[EXE]}
	\end{enumerate} 
\end{enumerate}

\subsubsection{Poisson CDF}
The cumulative probability $\textrm{P}(X \leq x)$ can be calculated by using the in-built \code{\addtoindex{PoissonCD}} function:
\begin{center}
	\code{PoissonCD(x,$\pilambdaup$)}
\end{center}

Alternatively, following instructions in the previous section (Section \ref{sec:poisson} $\Rightarrow$ Poisson PDF), and choosing \code{Pcd} instead. 

\subsubsection{Additive Property of Poisson Random Variable}
If $X \sim \textrm{Po}(\pilambdaup)$ and $Y \sim \textrm{Po}(\mu)$, where $X$ and $Y$ are \textbf{independent}, then:
\begin{equation}
	X+Y \sim \textrm{Po}(\pilambdaup + \mu)
\end{equation}

\section{Continuous Random Variable}
The probability for a continuous variable $X$ to fall within a particular region $[a,b]$ is given by:
\begin{equation}
	\mathrm{P}(a \leq X \leq b) = \int_{a}^{b} f(x)
\end{equation}
where $f(x)$ is the \textit{probability density function} (PDF) of X.

In particular,
\[
\int_{-\infty}^{\infty}f(x) = 1
\]

Moreover, 
\begin{align*}
	\mathrm{P}(a \leq X \leq b) &= \mathrm{P}(a \leq X < b)\\
	&= \mathrm{P}(a < X \leq b) \\
	&= \mathrm{P}(a < X < b)
\end{align*}

\subsection{Normal Distribution}
A (continuous) random variable $X\in \mathbb{R}$ that follows a normal distribution with mean $\mu$ and standard deviation $\sigma$ has a \textit{probability density function} (PDF) given by:

\begin{equation}
	f(x)=\frac{1}{\sigma\sqrt{2\pi}} \cdot e^{\frac{-(x-\mu)^2}{2\sigma^2}}
\end{equation}

We write $X \sim \mathrm{N} (\mu,\sigma^2)$

\note{$Z$ is reserved to represent the standard normal distribution. Do not use it to represent your variables (that do not follow standard normal distribution).}

\subsubsection{Normal Distribution PDF}
There are one of 2 ways to plot a graph of the normal distribution PDF:

\begin{itemize}
	\item Plot the actual equation
	\item Use the in-built \code{NormPD} function
\end{itemize}

However, it must be noted that the \code{NormPD} plots slower than using the actual equation. Using \code{G-Solv} is also slower.

The usage of \code{\addtoindex{NormPD}} is 
\begin{center}
	\code{NormPD(X,[$\sigma$,$\mu$])}
\end{center}

In this case, \code{X} can be either a single value, or a \code{List}. The corresponding $p$-value will also be adjusted accordingly.

If $\sigma$ and $\mu$ are not provided, the standard normal distribution with $\sigma=1$ and $\mu=0$ is assumed.

\note{You should not use this command to calculate the probability of a certain random variable $\textrm{P}(X = x)$ where $x \in \mathbb{R}$. This is because for a continuous random variable, this does not make sense. It should always be a range. (i.e. use the CDF in the next section)}

\subsubsection{Normal Distribution CDF}
CDF stands for \textit{Cumulative Distribution Function}. This can be calculated by taking the integral of the normal PDF from $-\infty$ to $x$. One can take integral by plotting the graph out (refer to previous section), and then \code{G-Solv > $\int$dx}.

Alternatively, you can use the built-in \code{\addtoindex{NormCD}}:
\begin{center}
	\code{NormCD(<Lower>,<Upper>,[$\sigma$,$\mu$])}
\end{center}

If $\sigma$ and $\mu$ are not provided, the standard normal distribution with $\sigma=1$ and $\mu=0$ is assumed.

To plot the Normal Distribution CDF, you can use:
\begin{center}
	\code{Y = NormCD(-1E99,X,[$\sigma$,$\mu$])}
\end{center}

\note{\code{-1E99} is to simulate $-\infty$.}

Alternatively, we can use the \code{STAT} app to obtain the probability:
\begin{enumerate}
	\item Go into \code{STAT} (\code{2} in the menu)
	\item Press \code{DIST}\Ffive\code{> NORM}\Fone\code{> Ncd}\fone
	\item Set the following:
	\begin{center}
		\setlength{\tabcolsep}{10pt}
		\renewcommand{\arraystretch}{1.2}
		\begin{tabular}{|l|l|}
			\hline
			Data		& Variable \\
			\hline
			Lower		& \textit{Lower} \\
			\hline
			Upper		& \textit{Upper} \\
			\hline
			$\sigma$	& $\sigma$ \\
			\hline
			$\mu$		& $\mu$ \\
			\hline
			Save Res	& \code{None} \\
			\hline
		\end{tabular}
	\end{center}
	\item Press \code{[EXE]} or scroll to \code{Execute} and press \code{CALC}\Fone
	\item You can do further manipulation of the value obtained by:
	\begin{enumerate}
		\item Going to \runmat
		\item Go to \code{[VARS] > STAT}\Fthree \code{> RESLT}\Fsix\code{> DIST}\Fthree
		\item There you can find the $p$ value by pressing $p$ \Fone and then \code{[EXE]}
	\end{enumerate} 
\end{enumerate}
\note{You cannot use this method for plotting}

\subsubsection{Finding the value given the probability}
One sometimes need to find $a$ given $\textrm{P}(X < a) = b$ where $a,b \in \mathbb{R}$ and $b$ is the probability of $X$ being less than $a$. 

To do this, we need the \code{\addtoindex{InvNormCD}} function built into the calculator. The usage of \code{InvNormCD} is as follows:
\begin{center}
	\code{InvNormCD($b$,[$\sigma$,$\mu$])}
\end{center}

This will give you back $a$.

\section{Use for Experimental Data/SPA}
The calculator can be used to determine the equation of a best fit line/parabola etc. from a set of data from experiments. From the main menu, enter the \code{S-SHT} app (\code{4}). 

In the app, simply enter each set of data into each row. Enter the data on the same axis under the same column.

For example, consider the following experimental data for an experiment involving electrical current ($I$) and potential difference ($V$): 

\begin{center}
	\setlength{\tabcolsep}{10pt}
	\renewcommand{\arraystretch}{1.2}
	\begin{tabular}{|c|c|}
		\hline
		\textbf{V/V}	& \textbf{I/A} \\
		\hline
		0.7148			& 0.0588 \\
		\hline
		0.7372			& 0.0633 \\
		\hline
		0.7780			& 0.0752 \\
		\hline
		1.058			& 0.143 \\
		\hline
		1.200			& 0.170 \\
		\hline
		1.340			& 0.210 \\
		\hline
	\end{tabular}
\end{center}

Enter this set of data into the GC as follows:

\begin{center}
	\setlength{\tabcolsep}{10pt}
	\renewcommand{\arraystretch}{1.2}
	\begin{tabular}{|r|c|c|}
		\hline
		{\tiny SHEET}& \textbf{A}	& \textbf{B} \\
		\hline
		\textbf{1}	& 0.7148		& 0.0588 \\
		\hline
		\textbf{2}	& 0.7372		& 0.0633 \\
		\hline
		\textbf{3}	& 0.7780		& 0.0752 \\
		\hline
		\textbf{4}	& 1.058			& 0.143 \\
		\hline
		\textbf{5}	& 1.200			& 0.170 \\
		\hline
		\textbf{6}	& 1.340			& 0.210 \\
		\hline
	\end{tabular}
\end{center}

To plot the points onto a graph, first press the right arrow key ($\triangleright$, \code{F6}) and select the \code{GRPH} option \Fone.

Once in the \code{GRPH} menu, we will first configure the graph options. 
\begin{enumerate}
	\item Press \code{SET} \Fsix and ensure that \code{StatGraph1} is shown on the first line and the \code{Graph Type} is \code{Scatter}.
	\item Move to \code{XCellRange} and enter \code{B1:B6}. This means that the values from cells \code{B1} to \code{B6} inclusive are that for the horizontal $x$-axis.
	\item Move to \code{YCellRange} and enter \code{A1:A6}.
	\item Finally, ensure that the \code{Frequency} option is \code{1}.
	\begin{itemize}
		\item You may select any \code{Mark Type} that suits your personal preferences.
	\end{itemize}
	\item Press \code{EXIT} to leave the configuration menu and return to the spreadsheet.
\end{enumerate}

Now, press \code{GPH1} \Fone. A graph with markings corresponding to the points on the spreadsheet should be yield.

You calculator can find the equation of a best fit line or curve. In this case, the data is expected to be linearly related (and it is linearly related).

\begin{enumerate}
	\item Press \code{CALC} \Fone.
	\item There are now a number of options corresponding to different types of graphs and relationships such as $x^2$, $\ln$, and $e^x$. In this case, our data is linear, so select \code{X} \Ftwo.
	\item Select $\code{ax+b}$ \Fone.
	\item A table with various values is yield. The value $a$ corresponds to the gradient, and $b$ is the $y$-intercept. $r^2$ represents how closely the data fit to the line, with the value of $1$ corresponding to a perfect, ideal fit.
	\item To view the graph visually, select \code{DRAW} \Fsix. To generate $y$-values using the equation, select \code{COPY} \Ffive, and select one of the \code{Y1}/\code{Y2\dots} to copy to.
\end{enumerate}

\note{You will need to add an $\times$ between the gradient and the $x$, or it will cause the function to error out.}

The linear regression result can also be obtained using the concept of best approximations, covered in Section \ref{sec:bestapprox} under Linear Algebra.

\subsection{Plotting Statistics/Visualization} \label{sec:plotstat}
In this section, we would explore ways we can use the calculator to help us visualize statistics, just like how it would appear in our lecture notes. The steps here are purely provided as an example, and to expose you to these functions, and you should adapt accordingly for your own use.

\subsubsection{Binomial/Poisson Distribution}
To visualize a binomial distribution, let's first generate a list of values for a certain binomial distribution. For example, let's say that $X \sim \text{B}(50, \frac{1}{3})$:

We can generate a list of probabilities easily in \runmat:

\begin{center}
	\code{BinomialPD(50,$\mathtt{\frac{1}{3}}$) $\mathtt{\rightarrow}$ List 1}
\end{center}

Thereafter, go into \code{STAT} and you'll see the probabilities all listed out for you in the \code{List 1}:

\begin{figure}[h]
	\centering
	\includegraphics[width=0.5\textwidth]{binom50}
\end{figure}

We can then easily graph this result:
\begin{enumerate}
	\item Go to \code{GRPH} \Fone \code{>} \code{SET} \fsix
	\item Scroll to \code{Graph Type} and choose \code{Bar} ($\triangleright$ \code{>} \code{F3})
	\item Select:
	\begin{tabular}{| l | l |}
		\hline
		\code{Data1} 		& \code{List1} \\
		\hline
		\code{Stick Type} 	& \code{Length} \\
		\hline
	\end{tabular}
	\item Press \code{EXIT} and then \code{GPH1} \fone
\end{enumerate}

The graph should have been plotted. You can scroll, zoom in, or out to see more:

\begin{figure}[h]
	\centering
	\includegraphics[width=0.5\textwidth]{binom50plot}
\end{figure}

This should look like those graphs they have in your lecture notes.

\note{the x-coordinate of 1 corresponds to $\text{P}(X=0)$}

A similar procedure can be followed to plot a graph of a variable that follows a Poisson distribution.

\chapter{Linear Algebra} \label{chap:linalge}
Some of you might find yourself doing linear algebra, and needing to do matrix manipulations. In this chapter, I will outline some basic concepts covered in MA1101R (NUS H3 Course), and how you can use your GC to find the answer.

You may find certain concepts here useful for H2 Mathematics as well, especially in terms of matrix manipulations. I suggest you read this chapter as a complement of Chapter \ref{chap:vectors} (Vectors). 

\section{Storing Matrices}
accessing individual matrix elements

\section{Solving Linear Systems} \label{sec:solvLinSys}
introduce augemented matrix
how to augment if 2

\subsection{Elementary Row Operations and Row-Echelon Forms}
row equiv

\section{Best Approximations} \label{sec:bestapprox}

\cleardoublepage
% \addcontentsline{toc}{chapter}{Index}
\printindex

\end{document}